\documentclass{article}
\usepackage[utf8]{inputenc}
\usepackage[spanish]{babel}
\usepackage{listings}
\usepackage{graphicx}
\graphicspath{ {Images/} }
\usepackage{cite}

\begin{document}

\begin{titlepage}
    \begin{center}
        \vspace*{1cm}
            
        \Huge
        \textbf{Informe Escrito}
            
        \vspace{0.5cm}
        \LARGE
        Parcial 2 - Primera parte
            
        \vspace{1.5cm}
            
        \textbf{Julian Taborda Ramirez}
        
        \vspace{0.5cm}
        
        \textbf{Samuel Ruiz Vargas}
            
        \vfill
            
        \vspace{0.8cm}
            
        \Large
        Informatica II\\
        Universidad de Antioquia\\
        Medellín\\
        Septiembre de 2021
            
    \end{center}
\end{titlepage}

\tableofcontents
\vspace*{1.2cm}

\newpage

\section{Analisís del problema}
\label{analisis}

    \begin{flushleft}
    \subsection{Objetivo(s)}
    Nuestro objetivo es simple, necesitamos ser capaces de representar una imagen en una matriz de leds de un tamaño por determinar, de la forma más fiel posible e independiente de la resolución de la imagen, haciendo uso de las técnicas de submuestreo y sobremuestreo. 
    \end{flushleft}
    \vspace*{0.5cm}
    \begin{flushleft}
    \subsection{Herramientas}
    Como herramientas de hardware tenemos las tiras neopixel, la tarjeta Arduino uno R3. Además probamos las funciones que nos proporicionaron para analizar una imagen. 
    \end{flushleft}
    \vspace*{0.5cm}
    \begin{flushleft}
    \subsection{Problematica(s)}
     Una de las problematicas a la que nos enfrentamos es la forma en la que vamos a tratar la información extraida de la imagen en QT y posteriormente utilizar esa información en tinkercad.
     Aparte tenemos que pensar en como redimensionar la imagen de tal manera que se adapate a la matriz independientemente a su relación de aspecto.   
    \end{flushleft}
    \vspace*{2cm}
    
\newpage  
    
\section{Esquema de desarrollo algoritmico}
\label{esquema}
    \begin{flushleft}

    \end{flushleft}
\newpage
    
\section{Algoritmo implementado}
\label{implementado}
    \begin{flushleft}
    
    \subsection{Funciones}
    \begin{flushleft}
    
    \vspace*{0.3cm}
        
    \end{flushleft}
    \end{flushleft}
    
    \vspace*{0.1cm}
    
\section{Consideraciones}
\label{consideraciones}
    \begin{flushleft}
        
    \end{flushleft}
    \vspace*{2cm}
\newpage    
    

\vfill
\vspace*{0.5cm}
\bibliographystyle{IEEEtran}


\end{document}